\newpage
\section{Билет 7. Изменение количества движения конечного объема сплошной среды. Массовые и поверхностные силы. Вектор напряжения. Тензор напряжения. Механический смысл его компонент. Изменение момента количества движения}

\textit{\underline{Массовые силы}} - это силы распределённые по объему $V$. Пример: сила тяжести, гравитационные силы, силы инерции.

\textit{\underline{Поверхностные силы}} - это силы распределённые по поверхности сплошной среды. Например, если налить жидкость в стакан, то на поверхности $S$ соприкосновения жидкости со стенками сосуда будет наблюдаться силовое взаимодействие.

Если разделить объем $V$ сечением $S$ на два объема: $V_1$ и $V_2$, и если рассматривать движение одной из его частей, например $V_1$, то необходимо заменить $V_2$ массовыми силами, распределенными по $V_1$ и поверхностными, распределенными по $S$. Возьмем, некоторую точку $M$ внутри тела и рассмотрим в этой точке различные площадки $d\sigma$. Полную силу действующую со стороны $V_2$ на объем $V_1$ на площадке $d\sigma$ с нормалью $n$, обозначим через $d\vec{P}$. Пусть $d\vec{P} = \vec{p_n}d\sigma$. Такого рода поверхностные силы можно вводить в любой точке сплошной среды, они называются силами внутренних напряжений. 

В каждой такой точке $M$ сплошной среды существует бесконечно много векторов $\vec{p_n}$, соответствующих бесконечному набору площадок $d\sigma$. Однако между ними существует универсальная, не зависящая от частных свойств движущейся среды, связь:

\begin{center}
    \textit{\underline{Уравнение количества движения для конечного объема сплошной среды}}
\end{center}
$$
\frac{d}{dt}  \int_{V} \vec{v} \rho \,d\tau =  \int_{V} \vec{F} \rho \,dV  + \int_{\Sigma} \vec{p_n} \,d\sigma
$$

Выводится оно из уравнения количества движения для системы точек, которое обобщается из уравнения Ньютона для одной материальной точки:
$$
m \frac{d\vec{v}}{dt} = \frac{dm\vec{v}}{dt} = \vec{F}
$$
Если у нас система материальных точек:
$$
\frac{dm_i\vec{v_i}}{dt} = \vec{F_i}
$$
Где $F_i$ это все силы: и внешние, и внутренние. Тогда сложив эти уравнения получим:
$$
\sum_{i = 1}^{n} \frac{dm_i\vec{v_i}}{dt} = \sum_{i = 1}^{n} \vec{F^{(e)}_i}
$$
где справа стоит сумма только внешних сил, ибо внутренние попарно сократятся. 
$$
Q = \sum_{i = 1}^{n} m_i \vec{v_i} 
$$
называется количество движения системы. Обобщив это уравнение для конечного объема $V$, ограниченного поверхностью $\Sigma$, получим нужное нам уравнение количества движения для конечного объема сплошной среды:
$$
\frac{d}{dt}  \int_{V} \vec{v} \rho \,d\tau =  \int_{V} \vec{F} \rho \,dV  + \int_{\Sigma} \vec{p_n} \,d\sigma
$$
Вернёмся к вектору напряжений. Если рассмотреть уравнение количества движения для сколь угодно малого объема, то можно получить зависимость напряжения $\vec{p_n}$ на произвольной площадке $d\sigma$ от векторов напряжений $\vec{p^1}, \vec{p^2}, \vec{p^3}$ на координатных площадках:
$$
\vec{p_n} = \vec{p}^i\vec{n_i} = p^{ki}\vec{e_k}(\vec{e_i}\vec{n})
$$
Это равенство задаёт координаты \textit{\underline{тензора внутренних напряжений}} $P = p^{ki}\vec{e_k}\vec{e_i}$

\begin{center}
    \textit{\underline{Изменение момента количества движения}}
\end{center}
    Аналогично уравнению изменения количества движения, выводится и уравнение изменения момента количества движения.
    Умножив векторно слева на радиус-вектор $r$ получается уравнение моментов количества движения для одной материальной точки:
$$
\frac{d\vec{K}}{dt} = \vec{\mathfrak{M}}
$$
где $\vec{K} = [\vec{r} \times m\vec{v}]$ и $\vec{\mathfrak{M}} = [\vec{r} \times \vec{F}]$. Для системы получим:
$$
\frac{d\vec{K}}{dt} = \sum_{i=1}^{n}\vec{r_i}\times\vec{F_i^{(e)}}
$$
В классическом случае моментом количества движения объема V сплошной среды обычно называют 
$$
\vec{K} = \int_{V} (\vec{r} \times \vec{v}) \rho d \vec{\tau}
$$
И уравнение моментов количества движения запишется тогда как:
$$
\frac{d}{dt} \int_{V} (\vec{r} \times \vec{v}) \rho d\vec{\tau} = \int_{V} (\vec{r} \times \vec{F}) \rho d\vec{\tau} + \int_{\Sigma} (\vec{r} \times \vec{p_n}) \rho d\sigma
$$

Если на тело внешние силы не действуют, то очевидно, что момент количества движения постоянен